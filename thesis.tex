\documentclass[pdftex,12pt,a4paper]{report}
\usepackage[utf8]{inputenc} %koi8-r + cyrillic encoding
\usepackage[bulgarian]{babel}

\usepackage[pdftex]{graphicx}

\newcommand{\HRule}{\rule{\linewidth}{0.5mm}}

\begin{document}
\setcounter{secnumdepth}{3}
\setcounter{tocdepth}{3}
\begin{titlepage}
	\begin{center}
		{\Huge ТЕХНОЛОГИЧНО УЧИЛИЩЕ ЕЛЕКТРОННИ СИСТЕМИ\\[0.5cm]} {\Large към ТЕХНИЧЕСКИ УНИВЕРСИТЕТ - СОФИЯ\\[3cm]}
		{\Huge ДИПЛОМНА РАБОТА\\[3cm]}
		Тема: "Мултитенант" система за управление на интернет сайтове\\[3cm]
		\begin{minipage}{0.4\textwidth}
			\begin{flushleft} \large
				\emph{Дипломант:}\\
				Михаил \textsc{Здравков}
			\end{flushleft}
		\end{minipage}
		\begin{minipage}{0.4\textwidth}
			\begin{flushright} \large
				\emph{Научен ръководител:} \\
				Инж.~Кирил \textsc{Митов}
			\end{flushright}
		\end{minipage}

		\vfill

		% Bottom of the page
		{\large София 2014}

	\end{center}
\end{titlepage}
\tableofcontents
\pagebreak
\addcontentsline{toc}{part}{Увод}
{\Large\center Увод\\[1cm]}
\chapter[Първа глава]{}
\section[Преглед на подобни продукти] {Преглед на подобни продукти}
Системите за управление на съдържание (CMS) стават все по-достъпни за масовата публика, тъй като позволяват на незапознатия с технологиите потребител да създаде и управлява уебсайт без почти никакви специфични знания и подготовка. CMS технологията дава възможност на всеки да споделя информация в Интернет мрежата, без почти никакви усилия и средства. По информация от 2011г. WordPress — популярна CMS система — управлява около 22\% от всички новосъздадени уебсайтове в света. Това обяснява и бума на създаване на такъв софтуер — в Интернет могат да се намерят стотици CMS системи. От друга страна, с напредването на технологиите, уебсайтовете стават все по-интереактивни и видеото като медия започва да заема все по-значителна роля за предствянето на информацията в Интернет. За това говори поддръжката на видео таг от нови уеб технологиии, като HTML5 например. Има статистика от март 2013г, че всеки ден 100 милиона американци гледат видео в Интернет, което е с 43\% повече отколкото през 2010г. Очакванията са, че през 2014г. видеото ще заема 50\% от световния трафик в Интернет. От тези предпоставки следва да се увеличава и значението на системите за управление на съдържание, насочени към видео споделяне.
\subsection[Какви услуги предлагат] {Какви услуги предлагат}
В Интернет могат да се намерят множество системи за управление на съдържанието, които да са насочени към изграждане и управление на сайтове с видео съдържание. Проблемът при тях е, че ако въобще имат вграден модел за монетаризиране на сайта, то той е базиран главно на рекламно съдържание. В никоя от разгледаните от дипломанта CMS платформи не е вграден бизнес модел чрез абониране. (някои системи имат възможност за надграждане на тази функционалност — например vimp) Друга съществена разлика между останалите подобни проекти и нашия е това, че те се стремят по-скоро да позволяват създаване на сайтове подобни на YouTube, където крайните потребители да могат да споделят видео клипове, докато нашия се фокусира над споделяне само от администратора. Целта ни е платформата да предлата монетаризиране чрез абониране и това е причината да направим <ИМЕ>.
В уебсайтовете си повечето подобни продукти изброяват като функционалност неща като например:
\begin{itemize}
	\item Лесно качване на видеота от машината на потребителя;
	\item Интеграция на социални мрежи;
	\item Блокиране на коментарите под видео клип по желание;
	\item Управления на потребителските профили;
	\item Опция за изтриване на видео по всяко време;
\end{itemize}
Според мен всичко това е задължително да присъства във всяка сериозна CMS платформа и като потребител бих очаквал от платформата си да го поддържа без да го изтъква.
\subsubsection[Launchpad6]{Launchpad6}
Една от водещите CMS платоформи насочени към бързо и лесно изграждане на сайтове за видео споделяне е launchpad6 (www.launchpad6.com). Тази платформа има голям набор от вградена функционалност, като например предлагат: интеграция и за настолни и за мобилни устройства, съдържание генерирано от потребителя (User Generated Content — UGC), панел за модериране на UGC и др. Монетаризирането на уебсайтовете създадени чрез launchpad6 се осъществява с вграждането на рекламно съдържание. Главния недостатък на launchpad6 е, че най-евтиния пакет, съдържащ основната функционалност струва \$1000 на месец. Освен това launchpad6 не предлага бизнес модела, който ние предлагаме — абониране.
\subsubsection[Vimp]{Vimp}
Друга известна система е Vimp. (www.vimp.com) Vimp предлага няколко различни пакета, от безплатен пакет за некомерсиална употреба и най-основната функционалност, до различни бизнес и корпоративни планове, които добавят различна функционалност и възможности. Vimp предлага и интеграция на бизнес модел, чрез абониране, за което трябва да се плати начална такса от 349 евро. Ситстемата е разработена така, че да може да поддържа голямо количество клиенти онлайн. Други възможности, които Vimp предлага на клиентите си са например:
\begin{itemize}
	\item интеграция на системата в локална мрежа, вместо в глобалната такава;
	\item поддръжка на мобилни устройства;
	\item white-labeling;
	\item многоезичност;
	\item добавяне на допълнителни картинки, които да могат да се използват като минятюри на видео клипове;
	\item информация за потребители онлайн;
	\item плейър, който може да се вгражда в други уебсайтове (включително facebook интеграция);
	\item свързване чрез facebook;
	\item импортиране на видео съдържание от YouTube и Vimeo;
\end{itemize}
\subsubsection[На българския пазар] {На българския пазар}
Българския пазар не предлага нито една насочена към видео споделяне система  за управление на съдържанието. В интерес на истината успях да намеря една единствена българска CMS платформа изобщо - това е StenikCMS. (http://cms.stenik.bg) Тази платформа е предназначена главно към изработката на онлайн магазини, за това и предлага интеграция с различни видове складови и търговски софтуер.
\subsubsection[Заключение] {Заключение}
И трите разгледани проектa са по-подходящи за големи фирми или корпорации, които имат възможност да инвестират голяма сума при началото на работата си с платформата. За малки фирми или отделни хора, които не разполагат със съществен капитал, тези разгледани решения са неподходящи, тъй като те не предлагат сериозна възможност за започване с малък уебсайт и постепенно разрастване на плана, заедно с разстежа на потребителя.
\subsection[Какви технологии използват] {Какви технологии използват}
\subsubsection[StenikCMS] {StenikCMS}
\begin{itemize}
	\item StenikCMS е написан на програмния език PHP 5, използвайки софтуерната платформа symfony;
	\item За фронтовата част (форматиране и изглед) са избрани XHTML и CSS;
	\item За база данни е избрана MySQL;
	\item Използват също така и AJAX;
\end{itemize}
\subsubsection[Vimp] {Vimp}
\begin{itemize}
	\item Vimp, също като StenikCMS, е направен с PHP 5 и symfony;
	\item Предлага собствен хостинг в облака (предоставен от MIVITEC);
	\item HTML и CSS;
\end{itemize}
\subsubsection[Launchpad6] {Launchpad6}
\begin{itemize}
	\item Също написана на PHP;
	\item Използва Amazon Cloud Service за хостинг;
	\item HTML и CSS;
\end{itemize}
\subsection[Информация за използваните от тях технологии] {Информация за технологиите използвани от Vimp, Launchpad6 и StenikCMS}
\begin{itemize}
	\item PHP е един от най-използваните програмни езици за създаване на уеб приложения. Някои от най-известните уебсайтове, като Wikipedia, Facebook и Wordpress го използват. PHP е скриптов, императивен, обектно-ориентиран език, който работи върху сървърната страна.
	\item AJAX е съвкупност от технологии (HTML, CSS, XML, Javascript), чрез коитo посредством асинхронен обмен на малки порции от данни се осъществяват интерактивни уебсайтове. AJAX позволява да се променят части от уеб страница, без за целта тази страница да бъде презареждана от уеб браузъра.
	\item Изчисления в облака (cloud computing) е предоставянето на компютърни изчислителни услуги. Основните характеристики на изчисленията в облак, според изсвелдване на университета Бъркли в Калифорния са:
		\begin{itemize}
			\item Изчисленията в облак създават илюзията за безкрайни изчислителни ресурси, налични при поискване, с което се елиминира нуждата от правене на предварителни дългосрочни планове за доставка на такива ресурси.
			\item 	Елиминира се високата бариера за навлизане и се дава възможност на компаниите да започват с поръчката на малко хардуерни и системни ресурси и да ги увеличават само когато нараснат потребностите им.
			\item 	Облачните изчисления дават възможност да се заплащат само изконсумираните изчислителни ресурси.
		\end{itemize}
	\item Модел-изглед-контролер (MVC) е шаблон на архитектурен дизайн, който разделя бизнес логиката, изгледа и данните.
		\begin{itemize}
			\item Моделът представя данните, които искаме да обработваме, променяме, визуализирме и тн. Обикновенно това са някакви данни и обекти от реалния свят.
			\item Изгледа е частта от програмния код, която отговаря за описването на външния вид и дизайна, по които нашето приложение ще представя и визуализира информацията.
			\item Контролера е тази част от програмния код, която обработва моделите (данните) и осигурява връзка между изгледа и модела. Контролерът е това, което „върши нещо“.
			\item MVC модела дава предимства като например модуларност на приложението – всяка от съставните му части е отделена от другите и по този начин може да се обработва/модифицира независимо от другите части. Също така, MVC модела позволява лесно създаване на нови изгледи без да се променят моделите и контролерите.
			\item От друга страна всяко по-високо ниво на абстракция внася допълнителна сложност в софтуера.
		\end{itemize}
\end{itemize}
\subsection[Други технологии] {Други технологии, които се използват за създаването на уеб приложения}
\begin{itemize}
	\item ASP.NET е софтуерна платформа създадена от Microsoft, изградена въз основа на Common Language Runtime (CLR), което позволява на разработчиците да ползват .NET език по техен избор. Ние отхвърляме ASP.NET като евентуална технология за нашето приложение, тъй като не сме запознати с .NET платформата и защото предпочитаме да използваме безплатен софтуер с отворен код.
\end{itemize}

\chapter[Втора глава] {}
\section[Функционални изисквания] {Функционални изисквания}
\section[Програмен език и софтуерни средства] {Програмен език и софтуерни средства}
\end{document}
