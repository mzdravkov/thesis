\documentclass[pdftex,12pt,a4paper]{report}
\usepackage[utf8]{inputenc} %koi8-r + cyrillic encoding
\usepackage[bulgarian]{babel}

\usepackage[pdftex]{graphicx}
\usepackage{hyperref}
\hypersetup{
 	colorlinks = true
}

\newcommand{\HRule}{\rule{\linewidth}{0.5mm}}

\begin{document}
\setcounter{secnumdepth}{3}
\setcounter{tocdepth}{3}
\begin{titlepage}
	\begin{center}
		{\Huge ТЕХНОЛОГИЧНО УЧИЛИЩЕ ЕЛЕКТРОННИ СИСТЕМИ\\[0.5cm]} {\Large към ТЕХНИЧЕСКИ УНИВЕРСИТЕТ - СОФИЯ\\[3cm]}
		{\Huge ДИПЛОМНА РАБОТА\\[3cm]}
		Тема: "Мултитенант" система за управление на интернет сайтове\\[3cm]
		\begin{minipage}{0.4\textwidth}
			\begin{flushleft} \large
				\emph{Дипломант:}\\
				Михаил \textsc{Здравков}
			\end{flushleft}
		\end{minipage}
		\begin{minipage}{0.4\textwidth}
			\begin{flushright} \large
				\emph{Научен ръководител:} \\
				Инж.~Кирил \textsc{Митов}
			\end{flushright}
		\end{minipage}

		\vfill

		% Bottom of the page
		{\large София 2014}

	\end{center}
\end{titlepage}
\tableofcontents
\pagebreak
\addcontentsline{toc}{part}{Увод}
\chapter*{Увод}
Some увод
\chapter {Преглед на подобни продукти и някои технологии, които биха могли да бъдат използвани за направата на дипломната работа}
\section {Преглед на подобни продукти}
Някои приложения и решения доближаващи се по функционалност до разработваната дипломна работа (макар и само в някои аспекти):
\subsection {Heroku}
Heroku\footnote[1]{Източник на информацията за Heroku са \url{https://www.heroku.com} и \\\url{https://devcenter.heroku.com/articles/quickstart}} е платформа за хостване на Интернет приложения на облачно базиран сървър\footnote[2]{облачно базиран сървър е мрежа от компютри, които предоставят изчислителни ресурси на клиента.}, която може да бъде използвана за приложения, написани на редица програмни езици (някои от които са Ruby, Clojure, Python, Java и Scala). Целта на Heroku е да позволи на клиентите си да се фокусират над правенето на приложения, а не на инфраструктура. Управлението на приложенията се извършва чрез командния ред, а изпращането на програмния код чрез Git (популярна система за управление на версиите). Heroku позволява лесен разтеж на приложението - това става чрез система, която дава възможност да изберете на колко Dyno-та (Това е единица изчислителна мощ, която Heroku предлагат. Всяка една такава единица е лек (взима малко ресурси), изолиран контейнер, в който работи клиентското приложение). Всяко Dyno се равнянва на 512 мегабайта RAM памет и приоритет 1 в поделянето на процесорно време. С други думи, ако приложението на клиента изведнъж получи извънредно много заявки от клиенти, потребителя на Heroku може да увеличи броя Dyno-та, които използва и да получи повече ресурси от сървърите на Heroku, а когато количеството посещения на приложението му се нормализира, да намали броя Dyno-та, така че да не заплаща ресурси, които не използва.\\
Има обаче някои съществени разлики в поставените цели при Heroku и при разработваната дипломна работа, които водят до множество разлики при реализацията и при начина на употреба на двете системи. Heroku е създадена за да работи с различни приложения. Иначе казано, всеки клиент на системата може да качи своето приложение и то няма нищо общо с всички останали. Дипломната работа е създадена да работи с едно единствено приложение, което случи като шаблон за създаването на останалите. Разликата е, че Heroku е създадена за да помогне на програмиста да инсталира някъде своето приложение, докато дипломната работа дава възможност на всякакви потребители да получат свое копие на шаблонното приложение.
\section {Преглед на някои технологии, които биха могли да бъдат използвани за направата на дипломната работа}
\subsection {HTTP сървъри}
В тази секция ще разгледаме някои от популярните възможни приложения за предоставяне на съдържание в Интернет.
\subsubsection {Apache}
\subsubsection {nginx}
\subsubsection {IIS}
\subsubsection {GWS}
\subsection {Технологии за създавне на Интернет приложения}
\subsubsection {MVC}
\subsubsection {Multitenancy}
\subsubsection {HTML}
\subsubsection {CSS}
\subsubsection {JavaScript}
\subsubsection {PHP}
\subsubsection {Java Servlets}
\subsubsection {ASP.NET}
\subsubsection {Ruby on Rails}
\subsection {Технологии за изолиране на приложения}
\subsubsection {Виртуални машини}
\subsubsection {chroot}
\subsubsection {FreeBSD jail-ове}
\subsubsection {Docker}
\section {Заключение}
\chapter {Функционални изисквания към приложението. Аргументация на избора на развойните средства и среди. Описание на сценария на реализацията на продукта}
\section {Функционални изисквания}
\section {Програмен език и софтуерни средства}
\section {История на разработката на дипломната работа}
\subsection {Подход 1 - Multitenancy и Ruby on Rails}
\subsection {Подход 2 - Разделяне на клиентското приложение от дипломната работа и Ruby on Rails}
\subsection {Подход 3 - FreeBSD Jail-ове}
\subsection {Подход 4 - Docker}
%\bibliography{thesis}
%\bibliographystyle{plain}
\end{document}
