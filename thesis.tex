\documentclass[pdftex,12pt,a4paper]{report}
\usepackage[utf8]{inputenc} %koi8-r + cyrillic encoding
\usepackage[bulgarian]{babel}

\usepackage[pdftex]{graphicx}
\usepackage{hyperref}
\hypersetup{
 	colorlinks = true
}

\newcommand{\HRule}{\rule{\linewidth}{0.5mm}}

\begin{document}
\setcounter{secnumdepth}{3}
\setcounter{tocdepth}{3}
\begin{titlepage}
	\begin{center}
		{\Huge ТЕХНОЛОГИЧНО УЧИЛИЩЕ ЕЛЕКТРОННИ СИСТЕМИ\\[0.5cm]} {\Large към ТЕХНИЧЕСКИ УНИВЕРСИТЕТ - СОФИЯ\\[3cm]}
		{\Huge ДИПЛОМНА РАБОТА\\[3cm]}
		Тема: ''Мултитенант'' система за управление на интернет сайтове\\[3cm]
		\begin{minipage}{0.4\textwidth}
			\begin{flushleft} \large
				\emph{Дипломант:}\\
				Михаил \textsc{Здравков}
			\end{flushleft}
		\end{minipage}
		\begin{minipage}{0.4\textwidth}
			\begin{flushright} \large
				\emph{Научен ръководител:} \\
				Инж.~Кирил \textsc{Митов}
			\end{flushright}
		\end{minipage}

		\vfill

		% Bottom of the page
		{\large София 2014}

	\end{center}
\end{titlepage}
\tableofcontents
\pagebreak
\addcontentsline{toc}{part}{Увод}
\chapter*{Увод}
Some увод
\chapter {Преглед на подобни продукти и някои технологии, които биха могли да бъдат използвани за направата на дипломната работа}
\section {Преглед на подобни продукти}
Някои приложения и решения доближаващи се по функционалност до разработваната дипломна работа (макар и само в някои аспекти):
\subsection {Heroku}
Heroku\footnote{Източник на информацията за Heroku са \url{https://www.heroku.com} и \\\url{https://devcenter.heroku.com/articles/quickstart}} е платформа за хостване на Интернет приложения на облачно базиран сървър\footnote{облачно базиран сървър е мрежа от компютри, които предоставят изчислителни ресурси на клиента.}, която може да бъде използвана за приложения, написани на редица програмни езици (някои от които са Ruby, Clojure, Python, Java и Scala). Целта на Heroku е да позволи на клиентите си да се фокусират над правенето на приложения, а не на инфраструктура. Управлението на приложенията се извършва чрез командния ред, а изпращането на програмния код чрез Git (популярна система за управление на версиите). Heroku позволява лесен разтеж на приложението - това става чрез система, която дава възможност да изберете на колко Dyno-та (Това е единица изчислителна мощ, която Heroku предлагат. Всяка една такава единица е лек (взима малко ресурси), изолиран контейнер, в който работи клиентското приложение). Всяко Dyno се равнянва на 512 мегабайта RAM памет и приоритет 1 в поделянето на процесорно време. С други думи, ако приложението на клиента изведнъж получи извънредно много заявки от клиенти, потребителя на Heroku може да увеличи броя Dyno-та, които използва и да получи повече ресурси от сървърите на Heroku, а когато количеството посещения на приложението му се нормализира, да намали броя Dyno-та, така че да не заплаща ресурси, които не използва.\\
Има обаче някои съществени разлики в поставените цели при Heroku и при разработваната дипломна работа, които водят до множество разлики при реализацията и при начина на употреба на двете системи. Heroku е създадена за да работи с различни приложения. Иначе казано, всеки клиент на системата може да качи своето приложение и то няма нищо общо с всички останали. Дипломната работа е създадена да работи с едно единствено приложение, което случи като шаблон за създаването на останалите. Разликата е, че Heroku е създадена за да помогне на програмиста да инсталира някъде своето приложение, докато дипломната работа дава възможност на всякакви потребители да получат свое копие на шаблонното приложение.
\section {Преглед на някои технологии, които биха могли да бъдат използвани за направата на дипломната работа}
\subsection {HTTP сървъри}
В тази секция ще разгледаме някои от популярните възможни приложения за предоставяне на съдържание в Интернет.
\subsubsection {Apache}
Apache\footnote{Източник на информацията за Apache HTTP Server е \url{https://en.wikipedia.org/wiki/Apache_HTTP_Server}} HTTP Server или само Apache е уеб сървър с отворен код, който има ключова роля за първоначалното разрастване на WWW (World Wide Web). Чрез него работят над 70\% от сайтовете. Счита се от много специалисти за платформа, според която се разработват и оценяват другите уеб сървъри.
Приложението стартира на много операционни системи, включително Unix, GNU, FreeBSD, Linux, Solaris, Mac OS X, Microsoft Windows, OS/2, Novell NetWare и други платформи.
Apache се разработва от отворено общество от разработчици - Apache Software Foundation. Сървърът има възможности за промяна на съобщенията за грешки, удостоверяване на потребителите, договаряне на съдържанието (изключително полезно при многоезични сайтове), proxy модул, както и поддръжка на CGI и SSI. Има множество модули за Apache, които позволяват работа на разнообразни скриптове и осигуряване на динамично съдържание, криптиране, ограничаване и други.
\subsubsection {nginx}
nginx\footnote{Източник на информацията за nginx е \url{https://bg.wikipedia.org/wiki/Nginx}}е високопроизводителен уеб сървър и прокси под BSD лиценз. Подобно на други приложения от този вид, архитектурата на nginx е модулна - при компилиране на софтуера се определя кои модули да бъдат вградени в него. Съществуват и над 20 потребителски модула.
nginx може да се използва като обратен прокси сървър, който прехвърля всички или само определени заявки към други физически сървъри. Крайните сървъри могат се избират от nginx на ротационен принцип, но решенията кой от тях да се използва могат да се взимат и чрез по-сложни алгоритми, благодарение на допълнителни модули. Често срещана употреба на приложението е за обработване на заявки за статично съдържание и прехвърляне на по-сложните заявки за динамично съдържание към по-сложен уеб сървър, например Apache. Въпреки това, nginx има пълна FastCGI поддръжка и може да изпълнява скриптове на всеки език за програмиране, който поддържа този стандарт. Софтуерът може да се използва и като SMTP, POP3 и IMAP прокси сървър.
При определени ситуации, особено при обслужване на заявки за статично съдържание, nginx е по-бърз и заема по-малко ресурси от конкурентния софтуер - Apache и lighttpd.
\subsubsection {IIS}
\subsubsection {GWS}
\subsection {Технологии за създавне на Интернет приложения}
\subsubsection {MVC}
Модел-Изглед-Контролер (Model-View-Controller или MVC) е архитектурен шаблон за дизайн (design pattern) в програмирането, основан на разделянето на бизнес логиката от графичния интерфейс и данните в дадено приложение.
\begin{itemize}
  \item Model е частта от програмния код, която представя данните от реалния свят, върху които работим и които сме моделирали. Често моделът служи за свързване с база данни. Основната бизнес логика свързана с обработката на данните се извършва в моделите.
  \item View е частта от програмния код, която описва как ще изглежда уеб страницата, която потребителя ще види. В нея се избягва да има програмна логика и данните, които тя показва се взимат от моделите посредством контролерите.
  \item Controller е частта от програмния код, която служи за връзка между моделите и изгледите. Тя се занимава с това да взима нужната информация от модела и да я предоставя на изгледа. В нея се извършват също дейности като уторизация на потребителите, обработка на параметри от HTTP заявки и пренасочване към други страници.
\end{itemize}
Макар MVC да добавя допълнително ниво на абстракция и нова сложност, то носи и значителни предимства. Модулярността позволява да направите различен интерфейс за същите модели, като промените единствено изгледа и евентуално контролера. Друга полза е това, че различни разработчици могат да се занимават единствено с областите, които са в тяхната специалност, например уеб дизайнера да работи само върху изгледите (без да има нужда да познава моделите и контролерите), а пък друг разработчик да работи единствено върху моделите и контролерите, без да има нужда да знае как информацията ще бъде представена в изгледите.
\subsubsection {Multitenancy}
Multitenancy\footnote{Източник на информацията за Multitenancy е \url{https://en.wikipedia.org/wiki/Multitenancy}}(буквално преведено - Множествено наемателство) е принцип в софтуерната архитектура, където една инстанция на компютърната програма обслужва множество клиенти и се грижи за виртуалното разделяне на данните между ''наемателите'' (потребителите). Принципа контрастира с multi-instance (много инстанции), където за всеки клиент има отделна инстанция на приложението. Смята се, че multitenancy принципа е важна част от Cloud computing технологията. Преимуществата на Multitenancy са по-малкото нужни ресурси (тъй като всяка инстанция на приложението би взимала някакво количество ресурси, докато при multitenancy инстанцията е само една) и по-лесната комуникация между ''наемателите'' (защото комуникацията се извършва вътре в самото приложение, а не между приложенията). Някои от недостатъците са по-трудната разработка и по-трудното разрастване на програмата при множество потребители.
\subsubsection {HTML}
HTML\footnote{Източник на информацията за HTML е \url{https://bg.wikipedia.org/wiki/HTML}}, съкращение от HyperText Markup Language — на български ''език за маркиране на хипертекст'', е основният маркиращ език за описание и дизайн на уеб страници. HTML е стандарт в Интернет, а правилата се определят от международния консорциум W3C. Описанието на документа става чрез специални елементи, наречени HTML елементи или маркери, които се състоят от етикети или тагове (HTML tags) и ъглови скоби (като например елемента <html>). HTML елементите са основната градивна единица на уеб страниците. Чрез тях се оформят отделните части от текста на една уеб страница, като заглавия, цитати, раздели, хипертекстови препратки и т.н. Най-често HTML елементите са групирани по двойки <h1> и </h1>.
В повечето случаи HTML кодът е написан в текстови файлове и се хоства на сървъри, свързани към Интернет. Тези файлове съдържат текстово съдържание с маркери - инструкции за браузъра за това как да се показва текстът. Предназначението на уеб браузърите е да могат да прочетат HTML документите и да ги превърнат в уеб страници. Браузърите не показват HTML таговете, а ги използват, за да интерпретират съдържанието на страницата.
\subsubsection {CSS}
CSS\footnote{Източник на информацията за CSS е \url{https://bg.wikipedia.org/wiki/CSS}}(Cascading Style Sheets) е език за описание на стилове - използва се основно за описване на представянето на документ, написан на език за маркиране. Най-често се използва заедно с HTML, но може да се приложи върху произволен XML документ. Официално спецификацията на CSS се поддържа от W3C (World Wide Web Consortium). Създаден първоначално като средство за разделяне на съдържанието от представянето му, днес той се използва основно за визуално оформление на HTML страници.
CSS позволява да се определя как да изглеждат елементите на една HTML страница - шрифтове, размери, цветове, фонове, и др. CSS кодът се състои от последователност от стилови правила, всяко от които представлява селектор, последван от свойства и стойности. Например в следния CSS код:
p {font-size: 9pt;} има едно правило. То се състои от селектора p и свойството font-size, на което е зададена стойност 9pt. Това правило ще направи размера на шрифта във всички параграфи 9 точки.
\subsubsection {JavaScript}
JavaScript\footnote{Източник на информацията за JavaScript е \url{https://bg.wikipedia.org/wiki/JavaScript}} е интерпретиран език за програмиране, разпространяван с повечето Уеб браузъри. Поддържа обектно-ориентиран и функционален стил на програмиране. Създаден е в Netscape през 1995-та. Най-често се прилага към HTML-а на Интернет страница с цел добавяне на функционалност и зареждане на данни. JavaScript е програмен език, който позволява динамична промяна на поведението на браузъра в рамките на дадена HTML страницата. JavaScript се зарежда, интерпретира и изпълнява от уеб браузъра, който му осигурява достъп до Обектния модел на браузъра. JavaScript функции могат да се свържат със събития на страницата (например: движение/натискане на мишката, клавиатурата или елемент от страницата, и други потребителски действия). Javascript е сред най-широко разпространените езици за програмиране в Интернет. Прието е JavaScript програмите да се наричат скриптове.
\subsubsection {PHP}
PHP е скриптов език, работещ върху сървърната (обслужваща) страна език с отворен код, който е проектиран за уеб програмиране и е широко използван за създаване на сървърни приложения и динамично уеб-съдържание. PHP е обектно ориентиран език със синтаксис подобен на езиците С и Perl.
\subsubsection {Java Servlets}
\subsubsection {ASP.NET}
\subsubsection {Ruby on Rails}
Ruby on Rails (често съкращавано като Rails или RoR) е популярна платформа за разработване на уеб-приложения, написана изцяло на програмния език Ruby, включваща в себе си множество реализирани шаблони за програмиране, сред които Model-View-Controller, ORM (Object Relational Mapping) и много други. Ruby on Rails има за цел да улесни и ускори начина на разработване на уеб-приложенията. Самата софтуерна рамка е с отворен код. Съществен елемент от философията на платформата е "Конвенция пред конфигурация" (Convention over configuration). Това означава, че Ruby on Rails се стреми да предостави възможно най-готова конфигурация за най-честия случай. Много неща, които в други платформи ще трябва да бъдат направени от програмистта, в Rails са направени по подразбиране.
\subsection {Технологии за изолиране на програми}
\subsubsection {Виртуални машини}
Виртуална машина е софтуерно базирана емулация на компютър. Виртуалните машини се делят на два основни вида:
\begin{itemize}
  \item Системна виртуална машина предоставя цяла системна платформа, която да позволява изпълняването на цяла операционна система. Тези виртуални машини обикновено симулират съществуваща компютърна архитектура и са създадени за да позволят неща като:
  \begin{itemize}
    \item Пускане на програми на хардуер, който не е достъпен за използване. (например за изпълнение на програми върху остаряла и излязла от употреба компютърна архитектура)
    \item Създаването на множество инстанции на виртуалната машина, което води до по-ефективно използване на компютърните ресурси. (например във фирми или учебни заведения, където множество служители или ученици използват отделна операционна система, без да е нужно за всеки един да има отделен хардуер. По този начин може да се направи оптимизация на използваните ресурси, като виртуалните машини се разпределят на по-малко или повече реални компютри в зависимост от натовареността.)
  \end{itemize}
\item Процес-вирутална машина е създадена за да съдържа една единствена програма, което значи, че поддържа един единствен процес. Такива виртуални машини са обикновено свързани с определен програмен език или няколко такива и имат функцията да предоставят портативност на програмите между различните компютърни архитектури.
\end{itemize}
Важна характеристика на виртуалните машини е, че софтуера работещ във виртуалната машина е ограничен от ресурсите и ограниченията, които виртуалната машина налага и, че не може да напусне виртуалната среда, в която се намира. Виртуалните машини имат важно значение за Cloud computing технологията и имат широка употреба в множество сфери. Въпреки това, съществен недостатък при тази технология е значителното количество ресурси, което виртуалната машина заема.
\subsubsection {chroot}
chroot в Unix операционните системи е операция, която променя root директорията на даден процес и неговите наследници. Процес, работещ в такава модифицирана среда, не може да именова (и следователно, обикновено да достъпва) файлове извън своето файлово дърво. Модифицираната среда се нарича chroot jail. (chroot jail) chroot механизъма не е създаден за да предотвратява атаки от привилегировани потребители. На повечето системи привилегирован потребител може да направи втори chroot за да "пробие" изолираната среда, в която се намира.
\subsubsection {FreeBSD jail-ове}
\subsubsection {Docker}
\section {Заключение}
\chapter {Функционални изисквания към приложението. Аргументация на избора на развойните средства и среди. Описание на сценария на реализацията на продукта}
\section {Функционални изисквания}
\section {Програмен език и софтуерни средства}
\section {История на разработката на дипломната работа}
\subsection {Подход 1 - Multitenancy и Ruby on Rails}
\subsection {Подход 2 - Разделяне на клиентското приложение от дипломната работа и Ruby on Rails}
\subsection {Подход 3 - FreeBSD Jail-ове}
\subsection {Подход 4 - Docker}
%\bibliography{thesis}
%\bibliographystyle{plain}
\end{document}
